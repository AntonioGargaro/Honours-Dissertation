% !TEX root = ../thesis-example.tex
%
\chapter{Requirements}
\label{sec:requirements}

%\cleanchapterquote{Innovation distinguishes between a leader and a follower.}{Steve Jobs}{(CEO Apple Inc.)}

This section discusses requirements and evaluates them to set a clear scope for the project to achieve. The aims and objectives discussed in section \ref{sec:intro:aims} and \ref{sec:intro:objectives} are the bases of the requirements stated below. As a technical project, I will evaluate the functional and non-functional requirements.



\section{Non-Functional Requirements}
\label{sec:requirements:non_func}


\begin{table}[htb!]
\centering
\begin{tabular}{|l|p{0.6\textwidth}|p{0.1\textwidth}|l|}
\hline
\textbf{ID} & \textbf{Description} & \textbf{Priority} & \textbf{Status} \\ \hline\hline
NFR-1 & Sum of API fetch weights per second must be equal or less than the limit placed by Binance & Must & - \\ \hline
NFR-2 & Fetch chosen coin's real-time market data at least once per second & Must & - \\ \hline
NFR-3 & Perform process in figure \ref{fig:related:tradeprocess} in less than 1 second of retrieving data & Must & - \\ \hline
NFR-4 & Trade signals lead to successful trade execution within 15 seconds & Must & - \\\hline
NFR-5 & Web app front end must work on Google Chrome v.70 & Must & - \\\hline
NFR-6 & Data stored locally should not exceed 5MB per coin pair & Should & - \\ \hline
NFR-7 & Web app runs simultaneous user operations & Won't have this time & - \\\hline
NFR-8 & Always have an active internet connection & Must & - \\\hline
\end{tabular}
\caption{Non-Functional Requirements}
\label{table:requirements:non_func}
\end{table}

\subsection{Non-Functional Requirements Evaluation}
\noindent The Binance API \cite{WEB:BINANCE_API:2018} has set weightings for fetch requests that accumulate between time intervals. This is to prevent constant spamming of their server, so failing to adhere to the set weightings within a time interval will ban the IP of a client from three minutes to three days. Use case 1 in section \ref{sec:requirements:non_func:usecases} describes how this may occur in extension 3.b. Blocked from the exchange incapacitates the bot by disrupting the market data retrieval process which hinders the ability to generate trade signals. To prevent this, non-functional requirement NFR-1 in table \ref{table:requirements:non_func} must be implemented to ensure the continued operation of the bot as shown in extension 3.a.
    
The alternatives to constant data fetching is shown in the technology variation list, specifically item 2.a. This allows for constant market data to be pushed to the bot instead, however this has its downsides. It is set at the interval of 1 second \cite{WEB:BINANCE_API:2018} which, in particular scenarios, may not provided enough new information. Finding the right balance between these two technologies will require research in the development phase.

While considering NFR-1, ensuring the bot also has the most up-to-date market data for the coin being traded is imperative to providing correct analysis. This will be considerably important during times of high volatility to find the best entry price. Thus, NFR-2 states a minimum of one call every second, which is consistent to the web socket push interval discussed above. However, if this interval lags and fails to push, a fetch call must execute to receive this data. As discussed in the literature review (See section \ref{sec:related:cryptoAndTheirMarkets}), these API weightings are extremely restrictive in comparison to the stock market. Hence, NFR-2 would likely differ in a HFT bot development requirement for the equity market. 

The trading bot must be able to identify trading signals in a small enough time frame, before more market data is retrieved from the API. As seen by use case 2 in section \ref{sec:requirements:non_func:usecases}, a backlog of data that needs to be analysed would severely delay the bot's operations. Extension scenario 2.a results in the bot cancelling its operations and shutting down as it cannot evaluate fast enough. This would be detrimental to success of this project so NFR-3 is defined as a must.  This is the main theme behind the success of HFT (See section \ref{sec:related:algoTrading:HFT}) in equity markets. This will also contribute towards the fulfilment of generating positive returns in this projects aims by allowing the bot to stay competitive in the market.

Ensuring the bot can execute upon the trade signal in a timely manner allows entry into the market close to the signal's level. links NFR-4 to one of the project objectives. This will allow the user to monitor the bots operations, so they themselves can make their own judgement on its decisions. NFR-4 is the backbone for FR-1, FR-2, and FR-3 in table \ref{table:requirements:func} and so is a must for this project. Both NFR-5 and NFR-6 ensure quality and efficiency in the front and back ends respectively. Specifically, NFR-6 can see large amounts of data when multiple intervals are used and so data management clean ups may be necessary.

While the exact goals of this project is to develop a web app for tradings bots as SaaS, this is outwith the scope of this dissertation. Hence, as reference for future iterations, NFR-7 has been noted. However, NFR-8 prepares for the future by ensuring can always be accessed remotely.



\section{Functional Requirements}
\label{sec:requirements:unc}



\begin{table}[htb!]
\centering
\begin{tabular}{|l|p{0.65\textwidth}|l|l|}
\hline
\textbf{ID} & \textbf{Description} & \textbf{Priority} & \textbf{Status} \\ \hline\hline
FR-1 & The user can stop the bot and cancel trades at any time & Must & - \\ \hline
FR-2 & A detailed output about the bots current operations to be displayed to the user & Must & - \\ \hline
FR-3 & Web app must display coin's market data on a candlestick chart & Must & - \\\hline
FR-4 & Candlestick chart displaying buy and sell signals that are generated & Must & - \\\hline
FR-5 & Option to display back testing results to user & Must & - \\ \hline
FR-6 & User configuration of algorithm options & Should & - \\ \hline
FR-7 & Web app should display technical indicators bot is using on candlestick chart & Could & - \\\hline
FR-8 & Web app to display order book data & Could & - \\\hline
FR-9 & User tracking of coin selection on server & Could & - \\ \hline
FR-10 & Bot to generate a trade signal from multiple technical indicators & Must & - \\ \hline
\end{tabular}
\caption{Functional Requirements}
\label{table:requirements:func}
\end{table}


\subsection{Functional Requirements Evaluation}
\noindent 

\noindent To ensure a satisfactory user experience, prioritising control of the bot to the user is a must. Some users may want to follow the bots process closely and prevent trades, thus FR-1 in table \ref{table:requirements:func} is a must for this project. As discussed in use case 2 in section \ref{sec:requirements:func:usecases}, this process essentially cancels all pending trades on the Binance exchange and seizes technical analysis operations for the user. As seen by the extension use cases 4.a and 4.b for use case 2, multiple pending order cancellations will be attempted if the proper acknowledgement isn't received. Otherwise, the user will be warned to cancel the trades themselves.

FR-2, FR-3, and FR-4 continue to ensure a satisfactory user experience. These requirements are set to keep the user informed of current market conditions and bot operations. Use case 3 in section \ref{sec:requirements:func:usecases} describes how this operation may occur for market data of a specific coin already stored locally. This prevents additional latency contacting the Binance API and provides a more responsive experience. In the case that there is no data, or if its outdated, extension use case 3.a explains how the server will make appropriate data fetches from the Binance API and update the local data of the coin on the server.

FR-5 gives the user a performance evaluation of the bot, indicating if it can or cannot turn a profit. This can be tied together with FR-6 to provide customisability options, which may change the back testing results. Looking at use case 4 in section \ref{sec:requirements:func:usecases} steps through how this may be carried out by the server. Extension use case 6 shows that the user may clear the results to revert back to the default display. 

FR-7 and FR-8 would aid in keeping the user informed of the bots operations, providing evidence of trends in the market and the current status of liquidity. FR-9 tracks a users data request as they navigate through the front end UI. To ensure resources are not being wasted, web sockets are closed when no one is requesting a specific coins data. This system is considered for future iterations of this project out with the current scope of this project.  

FR-10 is the back bone to this project. Ensuring the bot can properly generate a trade signal that is likely to return a profit based off of the discussion in section \ref{sec:related:tradingStrategies}. This is the main aim of this project.


\section{Conclusion}
\label{sec:requirements:conclusion}
The requirements and their evaluation align with the aims of this project by providing analytics to the user while performing robust operations. The non-functional requirements ensure these operations are efficient, keeping the bot competitive while attempting to produce positive returns on investments, as per the aims of this project. The non-functional requirements tackle the objectives defined by the project, ensuring the user is always informed of the current operations, and displaying this data in a meaningful way on candlestick charts. 

Furthermore, the user can specify options for back-testing and real-time trading. This allows customisability options to create a unique trade style. This allows more control to the user, providing a greater user experience overall.

