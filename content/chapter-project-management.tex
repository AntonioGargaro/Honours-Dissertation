% !TEX root = ../thesis-example.tex
%
\chapter{Management}
\label{sec:management}

This chapter outlines the project as a whole, presenting a time line of tasks and milestones to achieve, the risks involved during development, and a discussion on issues that are relevant to the project.



\section{Project Plan}
\label{sec:management:pp}

\noindent Two Gantt charts were presented for both semesters this project is developed within. The first semester discusses how this project came into fruition, the time line of the literature review, and illustration of the planned research time on other sections. The second semester outlines a rough guide to the development of the project, important milestones to achieve, and the conclusion of this project.

\subsection{Semester 1}
\label{sec:management:pp:sem1}

\noindent Referring to the Gantt chart in appendix \ref{gantt:sem1} (Pg. \pageref{gantt:sem1}), the project plan displays the milestones for this semester. Specifically, the \textbf{Literature Review} was prioritised at the start of the semester ensuring plenty of time to research the AT space. I concluded that 5 weeks should be sufficient to cover most major areas in this area. While the \textbf{Literature Review Milestone} was completed, task \textbf{2.2} took considerably longer than my 5 week estimation due to rewrites. The rolling effect from this left less time than anticipated for other sections.

The requirements analysis was effected heavily by the overlap of task \textbf{2.2}, where I could not complete the task to a high level of quality without further cutting into the design and risk analysis stages. Thus, I decided to cut task \textbf{3.2} down, completing roughly 60\% of the original plan. This allowed tasks \textbf{4.1, 5.1, and 6.1} to be completed to a greater degree. I could then achieve task \textbf{4.2, 5.1, 6.2, 6.3, and 6.4} which concluded the \textbf{Deliverable 1 Milestone}. 



\subsection{Semester 2}
\label{sec:management:pp:sem2}

\noindent Referring to the Gantt chart in appendix \ref{gantt:sem2} (Pg. \pageref{gantt:sem2}), the development outline of the project is clearly displayed. I will be following an agile development method, creating plans for weekly sprints based on the SCRUM framework that will develop tasks \textbf{1.1 - 1.5}. 

This framework consists of a product backlog, sprint planning to create a sprint backlog, followed by progress reviews at the end of sprints. The tasks \textbf{1.1 - 1.5} will be the bases for my product backlog. The advantages of using this agile framework allows to focus attention on specific parts of the project and incrementally implement them. This will also allow for self-evaluation of the projects status, confirming deadlines are being met and that the project is on track for the milestone \textbf{Project Build 1.0}. 2 weeks is required to fully evaluate and test the project, as seen at task \textbf{1.6}. An important note from this project plan is the prioritisation of the back-end development first. As this is the core of the project, this will be the initial main focus.

All the work produced during the project development will be documented, as seen by task \textbf{2.1.} This task is vital to achieving the \textbf{Dissertation Draft Milestone}, and hence the \textbf{Dissertation Hand-In} and \textbf{Poster Session} milestones. To ensure I am on track for completing this task, I will present my documentation to my supervisor bi-weekly, and confirm the completion amount on the project time line. 


\section{Risk Management}
\label{sec:management:ra}

\noindent As risks towards the project can effect the quality of the final product, a risk analysis has been evaluated to prevent negative factors. This evaluation categorises each risks level of likelihood and the impact it would have on completing this project. Appropriate mitigating scenarios have been developed.

\subsection{Risk Identification and Analysis}

\noindent Referring to table \ref{table:management:risks}, risks have been identified and evaluated to present the chances of the risk occurring, and how much it will impact the project by. Low, Medium, and High enumerations represent ranges from,
\begin{enumerate}
    \item Low - 0\% $\leq$ n $\leq$ 33\%
    \item Medium - 33\% $\leq$ n $\leq$ 66\%
    \item High - 66\% $\leq$ n $\leq$ 100\%
\end{enumerate}

\begin{table}[htb!]
\centering
\begin{tabular}{|l|p{0.5\textwidth}|p{0.13\textwidth}|l|}
\hline
\textbf{ID} & \textbf{Description} & \textbf{Likelihood} & \textbf{Impact} \\ \hline\hline
R-1 & Other commitments interfering & Medium & Medium \\ \hline
R-2 & Binance API goes offline & Low & High \\ \hline
R-3 & A library dependency becomes unusable & Low & Medium \\ \hline
R-4 & Development time taking longer than expected & Medium & Medium \\ \hline
R-5 & Development library incompatibility issues & Low & Medium \\ \hline
R-6 & Loss of project sources & Low & Low \\ \hline
R-7 & Insufficient processing hardware for project  & Low & Medium \\ \hline
R-8 & Adding unnecessary features (gold plating) & Medium & Medium \\ \hline
\end{tabular}
\caption{Risk Table}
\label{table:management:risks}
\end{table}

\subsection{Risk Mitigation}

\noindent Referring to table \ref{table:management:risk_mitigation}, identifying resolutions to forecasted potential risks will minimise the project impact. Thus, mitigating risk scenarios have been developed.

\begin{table}[htb!]
\centering
\begin{tabular}{|l|p{0.78\textwidth}|}
\hline
\textbf{ID} & \textbf{Strategy} \\ \hline\hline
R-1 & Distribute work load appropriately around other commitments and ensure core functionality can be implemented. \\ \hline
R-2 & Either, swap to another high volume exchange available from McHardy\footnote{https://github.com/sammchardy} using similar wrapper design. Or, utilise the Coinbase API via their python wrapper\footnote{https://github.com/coinbase/coinbase-python}. \\ \hline
R-3 & Analyse why it is unusable and find another library that avoids the issue and can fulfil the project requirements. \\ \hline
R-4 & Analyse and reshuffle priorities to focus on developing as much of the core requirements. Reflect this in sprint reviews. \\ \hline
R-5 & Identify reasons for incompatibility and find suitable library replacements that resolve the issue. \\ \hline
R-6 & Develop project using version control on hosted servers. Make back-ups on regular intervals. \\ \hline
R-7 & Utilise powerful AWS servers on their free tier\footnote{https://aws.amazon.com/free/}. \\ \hline
R-8 & Confirm new features to be implemented meet the current sprint's backlog  \\ \hline
\end{tabular}
\caption{Risk Mitigation}
\label{table:management:risk_mitigation}
\end{table}


\section{Professional, Legal, Ethical and Social Issues}
\subsection{Professional Issues}
\noindent This project takes into consideration the code of conduct\footnote{https://www.bcs.org/upload/pdf/conduct.pdf} outlined by The British Computer Society, specifically pertaining transparency in the web app's operations as defined by \textbf{FR-2} in table \ref{table:requirements:func} (Pg. \pageref{table:requirements:func}). The project only looks to handle publicly available data distributed by the Binance API, and use licensed libraries (Pg. \pageref{sec:related:developmentLibraries}).

\subsection{Legal Issues}
\noindent As far as this project is concerned, the only legal issues is the use of properly licensed libraries. As such, licences will be displayed in the design section. There is little regulation for the crypto market as discussed in section \ref{sec:related:cryptoAndTheirMarkets} (Pg. \pageref{sec:related:cryptoAndTheirMarkets}).

\subsection{Ethical and Social Issues}
\noindent AT is a controversial topic amongst the equity and crypto market that has contributed towards flash crashes and the main players of some manipulative techniques. This project addresses these issues by trading the market with data analysis of trends \textbf{only.}  
