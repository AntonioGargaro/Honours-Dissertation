% !TEX root = ../thesis-example.tex
%
\chapter{Management}
\label{sec:management}

This section discusses the project as a whole, presenting a time line of tasks to complete, the risks involved during development, and discuss issues that are relevant to the project.



\section{Project Plan}
\label{sec:management:pp}

Two Gantt charts were presented for both semesters this project is developed within. The first semester discusses how this project came into fruition, the time lime of the literature review, and illustrating the of time taken on other sections. The second semester outlines a rough guide to the development of the project, import milestones to achieve, and the conclusion of this project.

\subsection{Semester 1}
\label{sec:management:pp:sem1}

\noindent Referring to the Gantt chart in section \ref{gantt:sem1} (Pg. \pageref{gantt:sem1}), the project plans displays the milestones for this semester. Specifically, the Literature Review was prioritised at the start of the semester ensuring I had time to concentrate on researching the AT space. I concluded that 5 weeks should be sufficient to ensure I touched on all areas of this field. While it was completed, task 2.2 took considerably longer than my 5 week estimation due to rewrites. The effect of this left less time than planned for other sections which are discussed below.

The requirements analysis was effected heavily by the overlap of task 2.2, where I could not complete any UML diagrams without further cutting into the design and risk analysis stages. Thus, I decided to cut task 3.2 short, completing roughly 60\% of the original plan. This allowed tasks \textbf{4.1, 5.1, and 7.1} to be completed to a greater degree.



\subsection{Semester 2}
\label{sec:management:pp:sem2}

\noindent Referring to the Gantt chart in section \ref{gantt:sem2} (Pg. \pageref{gantt:sem2}), the development outline of the project is clearly displayed. I will be following an agile development method, creating plans for weekly sprints based on the SCRUM framework that will match to tasks \textbf{1.1 to 1.5}. 

This framework consists of a product backlog, sprint planning to create a sprint backlog, followed by progress reviews at the end of sprints. The tasks \textbf{1.1 to 1.5} will be the bases for my product backlog. The advantages of using this agile framework allows to focus attention or specific parts of the project and incrementally implement them. This will also allow for self-evaluation of the projects status, confirming deadlines are being met and that project is on track for the milestone \textbf{Project Build 1.0}.


\section{Risk Analysis}
\label{sec:management:ra}



\section{Professional, Legal, Ethical and Social Issues}
\label{sec:management:plesi}



\subsection{Some References}
\label{sec:management:results:refs}
