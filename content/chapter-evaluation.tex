% !TEX root = ../thesis-example.tex
%
\chapter{Evaluation}
\label{sec:evaluation}

The evaluation of this project will be split into three categories. The usability and intuitiveness of the web app, the implementation of the trade process (See section \ref{sec:related:algoTrading:tradeprocess}), and server infrastructure implementation.


\section{Web App}
\label{sec:evaluation:ui}
\noindent A System Usability Scale\footnote{https://www.usability.gov/how-to-and-tools/methods/system-usability-scale.html} (SUS) will be undertaken on the final build of the project. As an industry standard evaluation method, it can establish the intuitiveness and approachability of the software. To carry out an in depth evaluation of this usability, I aim to test 15 persons. This is enough coverage to determine its usability.

Evaluation of the front end implementation will be compared towards the functional requirements \textbf{FR-1, FR-2, FR-3, FR-4, FR-5, FR-6, FR7, and FR-8} in table \ref{table:requirements:func}. As these requirements have been catered towards the aims and objectives, this will validate the completion of the project.


\section{Trade Process Implementation}
\label{sec:evaluation:tradeprocess}
\noindent Evaluation of the trade process implementation can be confirmed by our completion of requirements \textbf{NFR-3, NFR-4, and FR-10} in tables \ref{table:requirements:non_func} and \ref{table:requirements:func} (Pg. \pageref{table:requirements:non_func} and \pageref{table:requirements:func}). The analysis undertaken for this evaluation will be a review of the features implemented, provided by code snippets and related outputs.



\section{Server Infrastructure Implementation}
\label{sec:evaluation:simulteneous}
Evaluation of our server infrastructure implementation can be confirmed by our completion of requirements \textbf{NFR-1, NFR-2, NFR-6, and FR-9} in tables \ref{table:requirements:non_func} and \ref{table:requirements:func} (Pg. \pageref{table:requirements:non_func} and \pageref{table:requirements:func}). The analysis undertaken is equivalent to section \ref{sec:evaluation:tradeprocess}.