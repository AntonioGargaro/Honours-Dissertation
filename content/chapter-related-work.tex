% !TEX root = ../thesis-example.tex
%
\chapter{Related Work}
\label{sec:related}
Algorithmic trading (algo-trading) has been present since the mid 1990s with the rise of personal computer power and the internet \cite{WEB:PISANI:2010}. Applying this method of trading to the volatile and unregulated market of cryptocurrencies can yield extraordinary results, greater than what the stock market can offer. This chapter will discuss the topics Algorithmic Trading, Trading Strategies and Cryptocurrencies And Their Markets. These topics will provide understanding of how algorithms are used and why they have become a necessity to the operation of the stock market. Illustrating how these trading methods work will show why they can also be applied to cryptocurrencies. I will also cover varying types of trading strategies, show how effective they are and why they would be useful towards algo-trading in cryptocurrencies. Finally, I will introduce the world of cryptocurrencies and how their markets operate. I will discuss what makes the cryptocurrency market unique, compare against the tested stock market and explain how using the aforementioned trading methods can allow for it be exploited. 
% \cleanchapterquote{A picture is worth a thousand words. An interface is worth a thousand pictures.}{Ben Shneiderman}{(Professor for Computer Science)}


\section{Algorithmic Trading}
\label{sec:related:sec1}
Seth \cite{WEB:SETH:0001} states that algo-trading's largest portion of its trading cohort is associated to High-Frequency Trading (HFT). HFT is the primary trading method in the modern stock market and is one of the most discussed and controversial topics in financial trade. This method is characterised by the speed and volume of trades it can execute within a small time frame.



\section{Trading Strategies}
\label{sec:related:sec2}


\section{Cryptocurrencies And Their Markets}
\label{sec:related:sec3}


\section{Conclusion}
\label{sec:related:conclusion}

