% !TEX root = ../thesis-example.tex
%
\chapter{Related Work}
\label{sec:related}
Algorithmic trading (hereafter, referred to as algo-trading) has been present since the mid 1990s with the rise of personal computer power and the internet \cite{WEB:PISANI:2010}. Applying this method of trading to the volatile and unregulated market of cryptocurrencies can yield extraordinary results, greater than what the stock market can offer. This chapter will discuss the topics Algorithmic Trading, Trading Strategies and Cryptocurrencies And Their Markets. These topics will provide understanding of how algorithms are used and why they have become a necessity to the operation of the stock market. Illustrating how these trading methods work will show why they can also be applied to cryptocurrencies. I will also cover varying types of trading strategies, show how effective they are and why they would be useful towards algo-trading in cryptocurrencies. Finally, I will introduce the world of cryptocurrencies and how their markets operate. I will discuss what makes the cryptocurrency market unique, compare against the tested stock market and explain how using the aforementioned trading methods can allow for it be exploited. 
% \cleanchapterquote{A picture is worth a thousand words. An interface is worth a thousand pictures.}{Ben Shneiderman}{(Professor for Computer Science)}


\section{Algorithmic Trading}
\label{sec:related:algoTrading}
\noindent Seth \cite{WEB:SETH:0001} states that algo-trading's largest category of its trading cohort is associated to High-Frequency Trading (hereafter, referred to as HFT). HFT is the primary trading type used in the modern stock market and is one of the most discussed and controversial topics. This method is characterised by the speed and volume of trades it can execute within a small time frame.

Both Seth \cite{WEB:SETH:0001} and Chorida, Goyal, Lehmann and Saar \cite{REPORT:ChordiaEtAl:2013} state the fundamental requirement for a successful HFT algorithm is low latency - defined by Chorida et al as "strategies that respond to market events in the millisecond environment". Comparing human ability to analyse and react at this level provides the reason as to why HFT is so controversial, and why it has the edge in trading. While low latency is key to the stock market it is hardly available at the same scale in cryptocurrency markets. Bloomberg's Levine reports \cite{WEB:Levine:2018} that only recently updates coming from the largest US based crypto exchange - Coinbase\footnote{https://www.coinbase.com/} - are betting on becoming the first to support low latency HFT by offering colocation\footnote{Locating computers owned by trading firms inside the same area as the exchange's servers}. This scarcity of low latency communication between exchange and trader prevents any specific institution having the edge over another.

\noindent HFT offers a supplementary effect to the market with the added addition of increased liquidity. Chorida et al \cite{REPORT:ChordiaEtAl:2013} state that most of the liquidity in the stock market is provided by HFT algorithms based on their activities. Bajpai \cite{WEB:Bajpai:0001} explains that with increased liquidity by HFT allows for larger orders to be successful close to the current price and execute within a short period of time. He builds on this by stating that "liquidity is an important characteristic of a good market" while further explaining that liquidity greatly reduces the `bid-ask spread'. This is the price difference between the highest buy order and the lowest sell order. With a small (tight) spread the transaction costs are reduced by producing a smaller difference between the buy and sell prices. 

However, while Chorida et al \cite{REPORT:ChordiaEtAl:2013} agrees that when HFT operates correctly it can improve the quality of the market, it can also degrade it by becoming demanders of liquidity. The affect of this is increases to volatility with no market makers to meet the demand. Chorida et al state this is evident by the ``flash crash`` of May 2010 where HTF - while may not of triggered it - certainly affected price volatility. 
% compare goods and bads of volatility against the two points, head in a direction to show HFT volatility is good and that it outweighs these apparent market anomalies and conclude


\section{Trading Strategies}
\label{sec:related:tradingStrategies}


\section{Cryptocurrencies And Their Markets}
\label{sec:related:cryptoAndTheirMarkets}


\section{Conclusion}
\label{sec:related:conclusion}

