% !TEX root = ../thesis-example.tex
%
\chapter{Related Literature}
\label{sec:related}
Algorithmic trading (hereafter, referred to as AT) has been present since the mid 1990s with the rise of personal computer power and the internet \cite{WEB:PISANI:2010}. Applying this method of trading to the volatile and unregulated market of cryptocurrency (hereafter, referred to as crypto) can yield extraordinary results, with proprietary equity trading firms successfully capitalising on it. This chapter will discuss the topics Algorithmic Trading, Trading Strategies and Crypto And Their Markets. These topics will provide understanding of how algorithms are used and why they have become a necessity to the operation of the stock market. Illustrating how these trading methods work will show why they can also be applied to crypto. I will also cover varying types of trading strategies, show how effective they are and why they would be useful towards AT in crypto. Finally, I will introduce the world of crypto and how their markets operate. I will discuss what makes the crypto market unique, compare against the tested stock market and explain how using the aforementioned trading methods can allow for it be exploited. 
% \cleanchapterquote{A picture is worth a thousand words. An interface is worth a thousand pictures.}{Ben Shneiderman}{(Professor for Computer Science)}


\section{Algorithmic Trading}
\label{sec:related:algoTrading}
\noindent Treleaven, Falas and Lalchand \cite{ART:Treleaven:2013} define AT as "any form of trading using sophisticated algorithms (programmed systems) to automate all or some part of the trade cycle". AT fits into systematic trading, described as trading based on rules. This combines both slower trading using strategies and high frequency trading which is discussed in section \ref{sec:related:algoTrading:HFT}. Nuti, Mirghaemi, Treleaven (who is also in \cite{ART:Treleaven:2013}) and Yingsaeree \cite{ART:Nuti:2011} state AT was designed to automate trade while being profitable and executing orders optimally. It could also be used to disguise intentions of traders or take advantage of competitor strategies. These characteristics of AT has allowed it to fit into a wide range of use cases, which is why it is so widely used on the stock market, resulting in 50 to 60 percent of the total daily trade volume in the EU and US \cite{ART:Nuti:2011}. Analysing the process of how trades come to fruition will illustrate how this can be applied to this project.

\subsection{Trade Process}
\label{sec:related:algoTrading:tradeprocess}
The trade process for AT described by Nuti et al \cite{ART:Nuti:2011} and Treleaven et al  \cite{ART:Treleaven:2013} consists of data access, pretrade analysis, trading signal generation, trade execution and post-trade analysis. The steps displayed (figure \ref{fig:related:tradeprocess}) require heavy analysis of data to determine if the market conditions are correct for the entry or exit of an asset to turn a profit. The ultimate goal of AT is to gain profit or know when to cut losses. The research of this trading process by both Nuti et al \cite{ART:Nuti:2011} and Treleaven et al \cite{ART:Treleaven:2013} capture the fundamental steps required for an algorithm to work, considering all aspects of the market to trade on.

\begin{figure}[htb]
    \centering
	\includegraphics[width=0.55\textwidth]{content/graphics/AT-trade_process}
	\caption{Trade Process Figure by Nuti et al \cite{ART:Nuti:2011}: \textit{(a)} Research / Data, \textit{(b)} Pretrade Analysis, \textit{(c)} Trading Signal Generation; \textit{(d)} Trade Execution, \textit{(e)} Post-trade Analysis }
	\label{fig:related:tradeprocess}
\end{figure}

Treleaven et al \cite{ART:Treleaven:2013} discuss that access to clean data is crucial to the operation of AT. The specificity of data required depends on the type of algorithm being built. Treleaven et al describe data which can vary from financial, economic, social and news to real time, historic and analysed market data. The selection and validity of data can greatly affect the operation of AT and requires vetting. This is consistent with Chan's \cite{BOOK:Chan:2013}  discussion of using data to simulate strategies using backtesting, where the results of a strategy may appear positive, but performs poorly when applied to real time trading. This suggests that the data used in backtesting may be limited, failing to cover the scope of the market. Considering this, implementing a robust system on backtesting that covers various different data sets would improve the performance of an algorithm by preventing unforeseen issues.

Both Treleaven et al \cite{ART:Treleaven:2013} and Nuti et al \cite{ART:Nuti:2011} describe that pretrade analysis discerns trading opportunities from data with the aim of predicting future prices to generate trading signals. This stage has three main categories of analysis, fundamental\footnote{Information relating to markets to determine an asset's fair value e.g. interest rates}, technical\footnote{Historic, current and other market information to predict future prices} and quantitative\footnote{Mathematical models based on treating assets prices as random rather that deterministic (like technical analysis)}. As technical analysis is the aim of this project, I must consider the utilisation of trading strategies (see section \ref{sec:related:tradingStrategies}) to forecast prices based on trends and other market data. The use of three mathematical models: alpha, risk and transaction cost (figure \ref{fig:related:tradeprocess}, part b). Respectively, these models are used to predict future behaviour, evaluate levels of risk and determine potential costs induced. Determining these factors using all market information will allow for the best possible outcome of analysis. 

Trading signal generation differs from pretrade analysis, specifically determining the values the trade should execute with. Nuti et al \cite{ART:Nuti:2011} explains that this stage determines factors such as entry and exit strategies while considering risk involved. Nuti et al define two problems that could occur during entry, oscillating buy and sell signals and incorrect model assumptions. Failing to address these problems can result in high transaction costs or increased losses. Detecting when to exit by taking profits or minimising losses is crucial. By failing to address both these points could lead to an unprofitable and useless algorithm. The final point of risk management by Nuti et al is considered vital. Understanding the quantity of an asset to trade without adversely affecting risk can help to minimise losses on a bad signal and prevent unprofitable trades.

Finally, Nuti et al \cite{ART:Nuti:2011} states that trade execution determines the constraints a transaction can occur, such as costs or duration. The main point made at this stage is to not adversely affect the market with large orders, while executing the trade at the right moment. Deciding whether to split large orders up into smaller chunks could prevent a change of market momentum impacting the trade. However, obtaining the ideal price results in the speed at which you execute the trade. Nuti et al discuss that a market order\footnote{Buying/Selling an asset at the best possible price at the current time} could be considered, however this could reduce your profit percentage by buying/selling for more/less than by placing a limit order\footnote{Buying/Selling an asset for a set price}.

The trading process described by Nuti et al \cite{ART:Nuti:2011} and Treleaven et al \cite{ART:Treleaven:2013} correlates with how an algorithmic trading bot to cryptocurrencies could be implemented. By demonstrating research on and common pitfalls of these stages, a well designed trading bot can be implemented using technical analysis. While in depth analysis is provided at each step, not much is discussed about the actual implementation or success of these systems. This dissertation discusses how these steps can be applied to the nascent crypto market, using real data and providing results based on analytics. Although, this project looks to build a trading bot based on a form of slower pretrade analysis, it is not the most common trading method used in the stock market today. 

\subsection{High Frequency Trading}
\label{sec:related:algoTrading:HFT}
\noindent  Seth \cite{WEB:SETH:0001} states that AT's largest category of its trading cohort is associated to High-Frequency Trading (hereafter, referred to as HFT). While HFT techniques are not used in the development of this project, understanding there place in trade on the stock market is beneficial. HFT is the primary trading type in the modern stock market and is one of the most discussed and controversial topics. This method is characterised by the speed and volume of trades it can execute within a small time frame.

Both Seth \cite{WEB:SETH:0001} and Chorida, Goyal, Lehmann and Saar \cite{REPORT:ChordiaEtAl:2013} state the fundamental requirement for a successful HFT algorithm is low latency - defined by Chorida et al as "strategies that respond to market events in the millisecond environment". Comparing human ability to analyse and react at this level provides the reason as to why HFT is so controversial, and why it has the edge in trading. Treleaven et al \cite{ART:Treleaven:2013} makes the point that not much is known about how AT interacts with the market, leading to the events like the 2010 Flash Crash where HFT exacerbated it. The instability HFT provides to the stock market brings concern to regulators. However, low latency HFT is not possible in the crypto market (see section \ref{sec:related:cryptoAndTheirMarkets}) which attributes to the massive volatility the market experiences. Discussing the effects HFT has on the stock market is beneficial to evaluating the health of the market as whole and why the addition of HFT should be prioritised by crypto exchanges.

Chorida et al \cite{REPORT:ChordiaEtAl:2013} state that most of the liquidity in the stock market is provided by HFT algorithms based on their activities. Bajpai \cite{WEB:Bajpai:0001} explains that with increased liquidity, HFT allows for larger orders to be successful close to the current price and execute within a short period of time. He builds on this by stating that "liquidity is an important characteristic of a good market" while further explaining that liquidity greatly reduces the `bid-ask spread'. This is the price difference between the highest buy order and the lowest sell order. With a small (tight) spread the transaction costs are reduced by producing a smaller difference between the buy and sell prices. This incentivises multiple orders to be made consistently by traders, with little transactional cost. In turn, the market is provided with liquidity which ultimately reduces risk for investors. 

However, while Chorida et al \cite{REPORT:ChordiaEtAl:2013} agrees that when HFT operates correctly it can improve the quality of the market with liquidity, it can also degrade it by demanding liquidity without any market makers to fill this demand. This subsequently increases volatility by occurring a major shift in the assets price. Chorida et al state this is evident by the ``flash crash`` of May 2010 where HFT - while it may not of triggered it - certainly affected price volatility. A source from Anagnostidis \cite{UNPUB:Anagnostidis:2017} explains "the speed with which the quotes are posted and cancelled has been criticised by market participants because its creates a false sense of deep liquidity supply for a stock", giving partial insight as to why the market dipped so drastically. Anagnostidis summarises by stating that liquidity generated by submissions and cancellations does not translate into a persistent effect on liquidity supply. This false cushion of liquidity allowed the market to free fall causing the overreaction of selling during the flash crash of 2010. This greatly impacted the stability of the market as a whole and raised concerns about HFT's affect to regulators to prevent it from happening again \cite{WEB:Kaufman:2016}.   
%It can be argued that the algorithms were trying to minimise loss, showing they were performing correctly during the flash crash. These sudden reactions to market events would only be possible for HFT algorithms, and although the dip was attributed by cancellations and demand for liquidity, it kept investors losses as low as possible.

The sudden submission and cancellation of HFT is the basis of their characteristics, explains Anagnostidis \cite{UNPUB:Anagnostidis:2017}. This is vital to the success of spoofing strategies (See section \ref{sec:related:tradingStrategies}). Arnoldi \cite{JOURNAL:Arnoldi:2016} raises the point that the market can be manipulated by manipulating HFT algorithms through spoofing. He reviews his first case study on spoofing by arguing that this manipulation would only occur between algorithms and that human traders would be to clever to fall for this. He then asks whether algorithms or humans are to blame for these exploitative strategies. This raises the question if whether algorithms are not sophisticated enough to detect these manipulative strategies or that laws should be introduced to prevent this. But with the delayed deployment of low latency HFT in crypto (see section \ref{sec:related:cryptoAndTheirMarkets}), further work can be conducted in these areas.
%In his first case study he explains how a trader can fake order book density and liquidity by first, placing separate sell orders in with different prices - e.g. 1 order at 113.25 and 3 at 113.24. The lowest price listed is the current best ask price (lowest price for sell orders), adding 4 orders to the order book. Then by submitting a large buy order at his higher price of the 4 orders it cancels all his other orders (a trader can't trade with their self). This then shows a reduction of sell orders in the market, a bigger spread of the ask and bid prices and the current best bid price is the larger value of the 4 orders. Algorithms respond to these signals with buys, pushing the price up. 

However, this method of `submit-cancel` is also used for price discovery\footnote{Determining price of asset based on analysis of buyer and sellers} of an asset due to AT's precise market analysis. It provides informative predictions based on all market information available leading to readjustment of order prices. An empirical study by Brogaard, Hendershott and Riordan \cite{UNPUB:Brogaard:2017} concluded that HFT improves pricing efficiency\footnote{The assets price is best reflected by all information possessed}, making HFT contribute towards a healthy market. While Arnoldi \cite{JOURNAL:Arnoldi:2016} shows spoofing to be manipulative and Anagnostidis \cite{UNPUB:Anagnostidis:2017} raises the effect of fake liquidity provided by such strategies, Brogaard et al \cite{UNPUB:Brogaard:2017} infers towards the benefits of price discovery and effeciency HFT brings to the stock market. Ultimately, spoofing strategies and fake liquidity are just one of the many ways to `play' the market, but it would seem the pros by Brogaard et al outweigh these cons.

Looking at the totality of liquidity generated by HFT provides quality values to the market by tightening spreads, reducing transaction costs, leading to price discovery and strengthening price efficiency. While not all attributes of HFT are healthy towards the market, such as false liquidity and manipulative techniques, the market has shifted towards AT for the advantages these values have while also keeping firms competitive. Applying these values into the crypto market would add to the growing market's health as a whole by reducing volatility and providing liquidity. After illustrating how AT works and the benefits they can bring to the market, analysing trading strategies will provide insight of how they can be applied to the crypto market.


% compare goods and bads of volatility against the two points, head in a direction to show HFT volatility is good and that it outweighs these apparent market anomalies and conclude


\section{Trading Strategies}
\label{sec:related:tradingStrategies}


\section{Cryptocurrencies And Their Markets}
\label{sec:related:cryptoAndTheirMarkets}
While low latency millisecond transactions are essential to the stock market, the same methods would fail due to the inadequate latency times and heavy network restrictions set by crypto exchanges. Bloomberg's Levine reports \cite{WEB:Levine:2018} that only recently updates coming from the largest US based crypto exchange - Coinbase\footnote{https://www.coinbase.com/} - are betting on becoming the first to support low latency HFT by offering colocation\footnote{Locating computers owned by trading firms inside the same area as the exchange's servers}. This scarcity of low latency communication between exchange and trader prevents some trading strategies from being applied to the crypto market. The discussion of trading strategies suitable for crypto markets are discussed in section \ref{sec:related:tradingStrategies}.

However, HFT in the crypto market is still apparent despite missing the quoted low latency fundamental that Chorida et al \cite{REPORT:ChordiaEtAl:2013} suggest. Trading is still faster than what any human can achieve and analysis still occurs on the millisecond time frame, but server latency is sub par compared to the stock market. Meyer and Rennison \cite{ART:Meyer:2017} report that large proprietary HFT firms DRW, Jump Trading, DV Trading and Hehmeyer Trading have entered the crypto market. It would be clear to assume that DRW and others have only entered this space if their is profit to be made. This is due to some HFT strategies - such as arbitrage (See section \ref{sec:related:tradingStrategies}) - not requiring transactions to occur on the millisecond time frame. Other types of AT may also be used such as trend lines, technical analysis and market making (See section \ref{sec:related:tradingStrategies}). 

Thus, by using the suggested strategies as examples, or most likely some unreleased strategies that are kept secret, the crypto market has the potential for profit to be made. Furthermore, this suggests that low latency is also not a fundamental requirement for HFT to be successful in crypto markets and the use of other strategies can also be utilised. Literature on algorithms applying technical analysis are sparse, however how technical analysis can be applied towards both markets is discussed in section \ref{sec:related:tradingStrategies}.

\section{Conclusion}
\label{sec:related:conclusion}

