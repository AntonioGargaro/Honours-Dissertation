% !TEX root = ../thesis-example.tex
%
\chapter{Related Literature}
\label{sec:related}
Algorithmic trading (hereafter, referred to as algo-trading) has been present since the mid 1990s with the rise of personal computer power and the internet \cite{WEB:PISANI:2010}. Applying this method of trading to the volatile and unregulated market of cryptocurrency (hereafter, referred to as crypto) can yield extraordinary results, with proprietary equity trading firms successfully capitalising on it. This chapter will discuss the topics Algorithmic Trading, Trading Strategies and Crypto And Their Markets. These topics will provide understanding of how algorithms are used and why they have become a necessity to the operation of the stock market. Illustrating how these trading methods work will show why they can also be applied to crypto. I will also cover varying types of trading strategies, show how effective they are and why they would be useful towards algo-trading in crypto. Finally, I will introduce the world of crypto and how their markets operate. I will discuss what makes the crypto market unique, compare against the tested stock market and explain how using the aforementioned trading methods can allow for it be exploited. 
% \cleanchapterquote{A picture is worth a thousand words. An interface is worth a thousand pictures.}{Ben Shneiderman}{(Professor for Computer Science)}


\section{Algorithmic Trading}
\label{sec:related:algoTrading}
\noindent Seth \cite{WEB:SETH:0001} states that algo-trading's largest category of its trading cohort is associated to High-Frequency Trading (hereafter, referred to as HFT). HFT is the primary trading type used in the modern stock market and is one of the most discussed and controversial topics. This method is characterised by the speed and volume of trades it can execute within a small time frame.

Both Seth \cite{WEB:SETH:0001} and Chorida, Goyal, Lehmann and Saar \cite{REPORT:ChordiaEtAl:2013} state the fundamental requirement for a successful HFT algorithm is low latency - defined by Chorida et al as "strategies that respond to market events in the millisecond environment". Comparing human ability to analyse and react at this level provides the reason as to why HFT is so controversial, and why it has the edge in trading. While low latency millisecond transactions are essential to the stock market, the same methods would fail due to the inadequate latency times and heavy network restrictions set by crypto exchanges. Bloomberg's Levine reports \cite{WEB:Levine:2018} that only recently updates coming from the largest US based crypto exchange - Coinbase\footnote{https://www.coinbase.com/} - are betting on becoming the first to support low latency HFT by offering colocation\footnote{Locating computers owned by trading firms inside the same area as the exchange's servers}. This scarcity of low latency communication between exchange and trader prevents some trading strategies from being applied to the crypto market. 

However, HFT in the crypto market is still highly successful without the quoted low latency fundamental that Chorida et al suggest. Trading is still faster than what any human can achieve and analysis still occurs on the millisecond time frame, but server latency is sub par compared to the stock market. Meyer and Rennison \cite{ART:Meyer:2017} report that large proprietary HFT firms DRW, Jump Trading, DV Trading and Hehmeyer Trading have entered the crypto market. This is due to other HFT strategies - such as arbitrage, trend lines and market making (See section \ref{sec:related:tradingStrategies}) - do not require  transactions to occur on the millisecond time frame. It would be clear to assume that DRW and others have only entered this space if their is profit to be made. Thus, by using the suggested strategies as examples, or most likely some unreleased strategies that are kept secret, the crypto market has the potential for profit to be made. Furthermore, this suggests that low latency is also not a fundamental requirement for HFT to be successful in crypto markets.

\noindent HFT also offers a supplementary effect to the market with the added addition of increased liquidity. Chorida et al \cite{REPORT:ChordiaEtAl:2013} state that most of the liquidity in the stock market is provided by HFT algorithms based on their activities. Bajpai \cite{WEB:Bajpai:0001} explains that with increased liquidity, HFT allows for larger orders to be successful close to the current price and execute within a short period of time. He builds on this by stating that "liquidity is an important characteristic of a good market" while further explaining that liquidity greatly reduces the `bid-ask spread'. This is the price difference between the highest buy order and the lowest sell order. With a small (tight) spread the transaction costs are reduced by producing a smaller difference between the buy and sell prices. 

However, while Chorida et al \cite{REPORT:ChordiaEtAl:2013} agrees that when HFT operates correctly it can improve the quality of the market with liquidity, it can also degrade it by demanding liquidity without any market makers to fill this demand. This subsequently increases volatility by occurring a major shift in the assets price. Chorida et al state this is evident by the ``flash crash`` of May 2010 where HFT - while it may not of triggered it - certainly affected price volatility. A source from Anagnostidis \cite{UNPUB:Anagnostidis:2017} explains "the speed with which the quotes are posted and cancelled has been criticised by market participants because its creates a false sense of deep liquidity supply for a stock", giving partial insight as to why the market dipped so drastically. Anagnostidis summarises by stating that liquidity generated by submissions and cancellations does not translate into a persistent effect on liquidity supply. The fact that the algorithms were trying to minimise loss shows they were performing correctly during the flash crash. The battle of these algorithms attempting to undercut each other drove the price down drastically, but they ultimate achieved the best position they could during this dip. These sudden reactions to market events would only be possible for HFT algorithms, and although the dip was attributed by cancellations and demand for liquidity, it kept investors losses as low as possible.

Furthermore, the sudden submission and cancellation of HFT is the basis of their characteristics, explains Anagnostidis. This is vital to the success of spoofing strategies (See section \ref{sec:related:tradingStrategies}). While the flash crash was not purely affected by spoofing strategies, it defines the perfect results. This method of `submit-cancel` is also used for price discovery\footnote{Determining price of asset based on analysis of buyer and sellers} of an asset due to algo-trading's market analysis providing superior informative predictions. By understand interaction between buyers and sellers, and the state of the market, the algorithm can predict and readjust the pricing of it's order. An empirical study by Brogaard, Hendershott and Riordan \cite{UNPUB:Brogaard:2017} conclude that HFT improves pricing efficiency\footnote{The assets price is best reflected by all information possessed}. 
% compare goods and bads of volatility against the two points, head in a direction to show HFT volatility is good and that it outweighs these apparent market anomalies and conclude


\section{Trading Strategies}
\label{sec:related:tradingStrategies}


\section{Cryptocurrencies And Their Markets}
\label{sec:related:cryptoAndTheirMarkets}


\section{Conclusion}
\label{sec:related:conclusion}

