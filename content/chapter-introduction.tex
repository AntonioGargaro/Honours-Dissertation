% !TEX root = ../thesis-example.tex

% WHAT TO DO :
% Introduction: summarising objectives, problems solved to achieve the objectives, methods, results, achievements and limits, and sketching the organisation of the dissertation.
\chapter{Introduction}
\label{sec:intro}

% \cleanchapterquote{You can’t do better design with a computer, but you can speed up your work enormously.}{Wim Crouwel}{(Graphic designer and typographer)}

The boom of cryptocurrencies and the youth of their markets allows for a new possibility of volatile trading. However, capitalising on this market by trading on human emotions, impulses or availability can lead to error prone decisions or missed opportunities. A bot is entirely more capable of handling these issues that institutional trading on the US stock market accounted for 61\% or 6.73 billion shares \cite{WEB:Cheng:2017} per day by algorithmic trading in 2009. With constant availability and real-time analysis, bots can react fast to sudden market changes, long or short a currency and minimise loss through analysis of market data. Developing a web app presents a widely available remote interface to interact with this service, pushing the computational requirements to a server rather than the user's own hardware.






\section{Aims}
\label{sec:intro:aims}
\noindent To develop a web app that utilises technical analysis on data sets provided by the cryptocurrency exchange Binance, via their API \cite{WEB:BINANCE_API:2018}. A trading strategy to generate buy and sell signals based on the market's current trends will be derived from this data. This will be implemented as a web server as to interact with the bot from a web based dashboard. The signals generated and actions taken by the bot will be displayed to the user in a meaningful way.


\section{Objectives}
\label{sec:intro:objectives}
\noindent The bot shall be configurable to allow the user to customise how their assets are used, such as setting a certain percent or amount of a specified currency, e.g. 100\% of Bitcoin (BTC)\footnote{The ticker symbol to identify Bitcoin} holdings. It should also include options specific to the way the bot trades, such as minimum profit to take and time intervals\footnote{m = minute, h = hour, w = week, M = month e.g. [30m, 4h, 1w, 1M, ...]} to trade at.

Information should be displayed in a useful way to the user about what the bot is doing. Indicators and other useful identifiers will be displayed on a candlestick chart to make clear what the bot has done and is currently doing. A log system of the bots actions will be stored and pushed to the user.

Lastly, an option to back-test the bot's trading algorithm shall be available to the user. This shall present a way to evaluate the bot's performance before the user enters into live trades on the market. 



\section{Thesis Structure}
\label{sec:intro:structure}

\noindent\textbf{Chapter \ref{sec:related}} \\[0.2em]
I cover related literature on algorithmic trading, the basics of trend indicators, the fundamentals of cryptocurrencies and their markets, and relevant libraries.

\noindent\textbf{Chapter \ref{sec:requirements}} \\[0.2em]
I identify use case scenarios and the requirements of this project, showing appropriate analysis and justification where required.

\noindent\textbf{Chapter \ref{sec:design}} \\[0.2em]
I construct an initial design of the development stack based on various technologies that will be used.

\noindent\textbf{Chapter \ref{sec:evaluation}} \\[0.2em]
I discuss the evaluation methods of this project.

\noindent\textbf{Chapter \ref{sec:management}} \\[0.2em]
I outline steps taken to manage this project by providing time lines, justifications, and risk assessments.