% !TEX root = ../thesis-example.tex
%
\chapter{Design}
\label{sec:design}
\vspace{-4em}{

As discussed in section \ref{sec:related:developmentLibraries} (Pg. \pageref{sec:related:developmentLibraries}), the main libraries have been researched and evaluated. Below is the listed technology stack currently established.

\section{Front End Stack}
\label{sec:design:frontend}
\begin{enumerate}
    \item JavaScript (ES6)
    \item React JS  -   Library for builing UI
    \item Redux     -   Application State Management
    \item Websockets \& API Fetches - Network Communication Protocols
    \item Webpack   - JS Module Builder
    \item jQuery    - Simplify DOM Traversals
    \item Material-UI   - React UI Framework
\end{enumerate}

\section{Back End Stack}
\label{sec:design:motivation}
\begin{enumerate}
    \item Flask \& Flask RESTful API - Web server framework \& Communications
    \item Fask SQL Alchemy          - SQLite for user tracking
    \item Binance Python Wrapper    - Wrapper to contact Binance API
    \item Backtrader                - Backtesting library with multiple indicators
    \item pandas                    - Data structures and analysis
\end{enumerate}

\section{Evaluation}
\label{sec:design:eval}

\noindent The front end stack is consistent with modern day singe page web apps. Specifically, React is a powerful UI library with great control. Redux simplifies management of the entire application state by providing a central store. The back end stack provides powerful infrastructure and data processing tools. The development team has used most of these technologies prior to this project.


}